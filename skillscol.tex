% ----- Languages -----
\cvsection{Languages}
    %% Fork v1.6.5c: The sloppypar* environment is used to avoid tags overlapping with section width
    \begin{sloppypar*}
      \cvtags{Haskell, C, Agda, Coq}

      \divider

      \cvtags{Python, SML, OCaml, Rust, Erlang}

      \divider

      \cvtags{Z, CSP, TLA/TLA+}
    \end{sloppypar*}
% ----- Languages -----
% ----- Tech STACK -----
\cvsection{TECH Stack}
    \begin{sloppypar*}
        \cvtags{Stack, Cabal, QuickCheck, Template Haskell}
        \cvtags{Monad Transformers, Type Families, GADTs, RankNTypes}
        \cvtags{Combinatory Parsers}

        \divider

        \cvtags{ViM, XMonad, Hakyll, Git, \LaTeX, Markdown, GNU/Linux, Emacs (Doom), Org}
    \end{sloppypar*}
% ----- TECH STACK -----

% ----- LEARNING -----
% \cvsection{Learning}
%     \begin{sloppypar*}
%         \cvtags{???}
%         \medskip

%     \end{sloppypar*}
% ----- LEARNING -----

% ----- LANGUAGES -----
\cvsection{Speak}
    \cvlang{Spanish}{Native} {\(\mid\)}
    % \divider
    \cvlang{English}{Fluent}\\
    \divider
    \cvlang{Portuguese}{Basic}
    %% Yeah I didn't spend too much time making all the
    %% spacing consistent... sorry. Use \smallskip, \medskip,
    %% \bigskip, \vpsace etc to make ajustments.
% ----- LANGUAGES -----

% ----- REFERENCES -----
\cvsection{References}
    \cvref{Prof.\ César Sánches}{IMDEA Software Instute}{cesar.sanchez@imdea.org}
    \divider

    \cvref{Prof.\ Mauro Jaskelioff}{IOG?}{g.delta@business.com}
% ----- REFERENCES -----

% ----- MOST PROUD -----
% \cvsection{Most Proud of}

% \cvachievement{\faTrophy}{Fantastic Achievement}{and some details about it}\\
% \divider
% \cvachievement{\faHeartbeat}{Another achievement}{more details about it of course}\\
% \divider
% \cvachievement{\faHeartbeat}{Another achievement}{more details about it of course}
% ----- MOST PROUD -----

% \cvsection{A Day of My Life}

% Adapted from @Jake's answer from http://tex.stackexchange.com/a/82729/226
% \wheelchart{outer radius}{inner radius}{
% comma-separated list of value/text width/color/detail}
% \wheelchart{1.5cm}{0.5cm}{%
%   6/8em/accent!30/{Sleep,\\beautiful sleep},
%   3/8em/accent!40/Hopeful novelist by night,
%   8/8em/accent!60/Daytime job,
%   2/10em/accent/Sports and relaxation,
%   5/6em/accent!20/Spending time with family
% }

% use ONLY \newpage if you want to force a page break for
% ONLY the current column
\newpage

% ----- EDUCATION -----
\cvsection{Education}
    \cvevent{PhD in Informatics}{UNR}{04 2015 -- 04 2023}{Rosario, Argentina}
    \begin{itemize}
        \item Thesis: Effectful Improvement Theory.
        \item Summary: An improvement theory for programming languages with
                algebraic effects.
        \item Advisor: PhD. Mauro Jaskelioff
    \end{itemize}
    \divider

    \cvevent{Computer Science Degree}{UNR}{03 2008 -- 03 2015}{Rosario, Argentina}
    \begin{itemize}
        \item Title: Simulation of Parallel Programs in Haskell
        \item Summary: A graphic library to observe and study the ``parallel
                structure'' of parallel programs in Haskell.
        % \item Advisors: PhD. Mauro Jaskelioff \\ \hspace{3.1em} \& PhD. Exequiel Rivas
        \item GPA: 9.60
    \end{itemize}
% ----- EDUCATION -----

% ----- PROJECTS -----
\cvsection{Projects}
    \cveventpp{Termina}{}{2023 -- Ongoing}{\cvreference{\faGithub}{https://github.com/orgs/termina-lang}}
    \begin{itemize}
        \item Transpiler from Termina to C plus a RTOS runtime for embedded systems.
        \item Implemented in Haskell and runtime in C.
    \end{itemize}
    \divider

    \cveventp{Smart Multi Contract Interaction}{}{2021 -- 2022}{\cvreference{\faGitlab}{https://gitlab.com/user/repo}}
    \begin{itemize}
        \item Library implementing the computational model of Tezos Blockchain.
        \item Implemented different scheduling strategies.
        \item Designed to reason about the behaviour of several contracts
            interacting within a single transaction.
    \end{itemize}
    \divider

    \cveventpp{Haskell Lola}{}{2019 -- 2020}{\cvreference{\faGithub}{https://gitlab.com/user/repo}\cvreference{\faGlobe}{https://software.imdea.org/hlola/}}
    \begin{itemize}
        \item Designed the core structure implementing HLola.
        \item Implemented the main abstraction lifting generic types of Haskell
employed as data-theories in Lola.
    \end{itemize}
    \divider

    \cveventpp{QuickFuzz}{}{2016 -- 2017}{\cvreference{\faGithub}{https://github.com/CIFASIS/QuickFuzz}}
    \begin{itemize}
        \item Out-of-the-box fuzzy generation of arbitrary files.
        \item Implemented Haskell arbitrary instances for arbitrary data
                structures in Haskell.
    \end{itemize}
    \divider

    \cveventpp{MegaDeTh}{}{2016 -- 2017}{\cvreference{\faGithub}{https://github.com/CIFASIS/megadeth}}
    \begin{itemize}
        \item Aggressive and generic instance derivation for recursive and
            mutually dependent Haskell data-types.
        \item Implemented entirely in Template Haskell.
    \end{itemize}
% ----- PROJECTS -----
%%% Local Variables:
%%% TeX-master: "ceresa"
%%% End:
