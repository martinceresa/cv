%%%%%%%%%%%%%%%%%
% This is an sample CV template created using altacv.cls
% (v1.3, 10 May 2020) written by LianTze Lim (liantze@gmail.com). Now compiles with pdfLaTeX, XeLaTeX and LuaLaTeX.
% (v1.6.5c, 27 Jun 2023) forked by Nicolás Omar González Passerino (nicolas.passerino@gmail.com)
%
%% It may be distributed and/or modified under the
%% conditions of the LaTeX Project Public License, either version 1.3
%% of this license or (at your option) any later version.
%% The latest version of this license is in
%%    http://www.latex-project.org/lppl.txt
%% and version 1.3 or later is part of all distributions of LaTeX
%% version 2003/12/01 or later.
%%%%%%%%%%%%%%%%

%% If you need to pass whatever options to xcolor
\PassOptionsToPackage{dvipsnames}{xcolor}

%% Use the "normalphoto" option if you want a normal photo instead of cropped to a circle
% \documentclass[10pt,a4paper,normalphoto]{altacv}

%% Use the "darkmode" option if you want a color palette used to
\documentclass[10pt,a4paper,ragged2e,withhyper]{altacv}

%% AltaCV uses the fontawesome5 and academicons fonts
%% and packages.
%% See http://texdoc.net/pkg/fontawesome5 and http://texdoc.net/pkg/academicons for full list of symbols. You MUST compile with XeLaTeX or LuaLaTeX if you want to use academicons.

%% Fork v1.6.5c: Overwriting sloppy environment to ignore any spaces and be used to solve overlapping cvtags
\newenvironment{sloppypar*}{\sloppy\ignorespaces}{\par}

% Change the page layout if you need to
\geometry{left=1.2cm,right=1.2cm,top=1cm,bottom=1cm,columnsep=0.75cm}

% The paracol package lets you typeset columns of text in parallel
\usepackage{paracol}

% Change the font if you want to, depending on whether
% you're using pdflatex or xelatex/lualatex
\ifxetexorluatex
  % If using xelatex or lualatex:
  \setmainfont{Roboto Slab}
  \setsansfont{Lato}
  \renewcommand{\familydefault}{\sfdefault}
\else
  % If using pdflatex:
  \usepackage[rm]{roboto}
  \usepackage[defaultsans]{lato}
  % \usepackage{sourcesanspro}
  \renewcommand{\familydefault}{\sfdefault}
\fi

% Fork (before v1.6.5a): Change the color codes to test your personal variant on any mode
\ifdarkmode%
  \definecolor{PrimaryColor}{HTML}{C69749}
  \definecolor{SecondaryColor}{HTML}{D49B54}
  \definecolor{ThirdColor}{HTML}{1877E8}
  \definecolor{BodyColor}{HTML}{ABABAB}
  \definecolor{EmphasisColor}{HTML}{ABABAB}
  \definecolor{BackgroundColor}{HTML}{191919}
\else%
  \definecolor{PrimaryColor}{HTML}{001F5A}
  \definecolor{SecondaryColor}{HTML}{0039AC}
  \definecolor{ThirdColor}{HTML}{F3890B}
  \definecolor{BodyColor}{HTML}{666666}
  \definecolor{EmphasisColor}{HTML}{2E2E2E}
  \definecolor{BackgroundColor}{HTML}{E2E2E2}
\fi%

\colorlet{name}{PrimaryColor}
\colorlet{tagline}{SecondaryColor}
\colorlet{heading}{PrimaryColor}
\colorlet{headingrule}{ThirdColor}
\colorlet{subheading}{SecondaryColor}
\colorlet{accent}{SecondaryColor}
\colorlet{emphasis}{EmphasisColor}
\colorlet{body}{BodyColor}
\pagecolor{BackgroundColor}

% Change some fonts, if necessary
\renewcommand{\namefont}{\Huge\rmfamily\bfseries}
\renewcommand{\personalinfofont}{\small\bfseries}
\renewcommand{\cvsectionfont}{\LARGE\rmfamily\bfseries}
\renewcommand{\cvsubsectionfont}{\large\bfseries}

% Change the bullets for itemize and rating marker
% for \cvskill if you want to
\renewcommand{\itemmarker}{{\small\textbullet}}
\renewcommand{\ratingmarker}{\faCircle}

% \input{pubs-authoryear.cfg}
\input{pubs-num.cfg}
%% sample.bib contains your publications
\addbibresource{conferences.bib}
\addbibresource{journals.bib}

\begin{document}
    \name{Martín Arnaldo Ceresa }
    \tagline{Postdoc Researcher}
    %% You can add multiple photos on the left or right
    \photoL{4cm}{martin}

    \personalinfo{
        \email{martin.ceresa@imdea.org}\smallskip
        \phone{+34-684319338}
        \location{Madrid, Spain}\\
        \linkedin{martinceresa}
        \github{martinceresa}
        % \npm{npmUser}
        % \dev{devtoUser}
        \homepage{martinceresa.github.io}
        %\medium{nicolasomar}
        %% You MUST add the academicons option to \documentclass, then compile with LuaLaTeX or XeLaTeX, if you want to use \orcid or other academicons commands.
        \orcid{0000-0003-4691-5831}
        %% You can add your own arbtrary detail with
        %% \printinfo{symbol}{detail}[optional hyperlink prefix]
        % \printinfo{\faPaw}{Hey ho!}[https://example.com/]
        %% Or you can declare your own field with
        %% \NewInfoFiled{fieldname}{symbol}[optional hyperlink prefix] and use it:
        % \NewInfoField{gitlab}{\faGitlab}[https://gitlab.com/]
        % \gitlab{your_id}
    }
    
    \makecvheader
    %% Depending on your tastes, you may want to make fonts of itemize environments slightly smaller
    % \AtBeginEnvironment{itemize}{\small}
    
    %% Set the left/right column width ratio to 6:4.
    \columnratio{0.4}

    % Start a 2-column paracol. Both the left and right columns will automatically
    % break across pages if things get too long.
    \begin{paracol}{2}
      %% Left column
      % ----- Languages -----
\cvsection{Programming Languages}
    %% Fork v1.6.5c: The sloppypar* environment is used to avoid tags overlapping with section width
    \begin{sloppypar*}
      \cvtags{\textbf{Haskell}, \textbf{Coq}, C, Agda}

      \divider

      \cvtags{\textbf{Rust}, \textbf{Python}, SML, OCaml, Erlang}

      \divider

      \cvtags{Z, CSP, TLA/TLA+}
    \end{sloppypar*}
% ----- Languages -----
% ----- Tech STACK -----
\cvsection{TECH Stack}
    \begin{sloppypar*}
        \cvtags{Stack, Cabal, QuickCheck, Template Haskell}
        \cvtags{Monad Transformers, Type Families, GADTs, RankNTypes}\\
        \cvtags{Combinatory Parsers}

        \divider

        \cvtags{ViM, XMonad, Hakyll, Git, \LaTeX, Markdown, GNU/Linux, Emacs (Doom), Org}
    \end{sloppypar*}
% ----- TECH STACK -----

% ----- LEARNING -----
% \cvsection{Learning}
%     \begin{sloppypar*}
%         \cvtags{???}
%         \medskip

%     \end{sloppypar*}
% ----- LEARNING -----

% ----- LANGUAGES -----
\cvsection{Speak}
    \cvlang{Spanish}{Native} {\(\mid\)}
    % \divider
    \cvlang{English}{Fluent}\\
    \divider
    \cvlang{Portuguese}{Basic}
    %% Yeah I didn't spend too much time making all the
    %% spacing consistent... sorry. Use \smallskip, \medskip,
    %% \bigskip, \vpsace etc to make ajustments.
% ----- LANGUAGES -----

% ----- REFERENCES -----
\cvsection{Referees}
    \cvrefWeb{Prof.\ César Sánches}{https://software.imdea.org/~cesar/}{IMDEA Software Instute}{cesar.sanchez@imdea.org}
    \divider

    \cvref{Prof.\ Mauro Jaskelioff}{IOG?}{g.delta@business.com}
% ----- REFERENCES -----

% ----- MOST PROUD -----
% \cvsection{Most Proud of}

% \cvachievement{\faTrophy}{Fantastic Achievement}{and some details about it}\\
% \divider
% \cvachievement{\faHeartbeat}{Another achievement}{more details about it of course}\\
% \divider
% \cvachievement{\faHeartbeat}{Another achievement}{more details about it of course}
% ----- MOST PROUD -----

% \cvsection{A Day of My Life}

% Adapted from @Jake's answer from http://tex.stackexchange.com/a/82729/226
% \wheelchart{outer radius}{inner radius}{
% comma-separated list of value/text width/color/detail}
% \wheelchart{1.5cm}{0.5cm}{%
%   6/8em/accent!30/{Sleep,\\beautiful sleep},
%   3/8em/accent!40/Hopeful novelist by night,
%   8/8em/accent!60/Daytime job,
%   2/10em/accent/Sports and relaxation,
%   5/6em/accent!20/Spending time with family
% }

% use ONLY \newpage if you want to force a page break for
% ONLY the current column
\newpage

% ----- EDUCATION -----
\cvsection{Education}
    \cvevent{PhD in Informatics}{UNR}{04 2015 -- 04 2023}{Rosario, Argentina}
    \begin{itemize}
        \item Thesis: Effectful Improvement Theory.
        \item Summary: An improvement theory for programming languages with
                algebraic effects.
        \item Advisor: PhD. Mauro Jaskelioff
    \end{itemize}
    \divider

    \cvevent{Computer Science Degree}{UNR}{03 2008 -- 03 2015}{Rosario, Argentina}
    \begin{itemize}
        \item Title: Simulation of Parallel Programs in Haskell
        \item Summary: A graphic library to observe and study the ``parallel
                structure'' of parallel programs in Haskell.
        % \item Advisors: PhD. Mauro Jaskelioff \\ \hspace{3.1em} \& PhD. Exequiel Rivas
        \item GPA: 9.60
    \end{itemize}
% ----- EDUCATION -----

% ----- PROJECTS -----
\cvsection{Projects}
    \cveventpp{Termina}{}{2023 -- Ongoing}{\cvreference{\faGithub}{https://github.com/orgs/termina-lang}}
    Transpiler from Termina to C plus a RTOS runtime for embedded systems.
    Designed and implemented in Haskell and runtime in C.
    \divider

    \cveventp{Smart Multi Contract Interaction}{}{2021 -- 2022}{\cvreference{\faGitlab}{https://gitlab.com/user/repo}}
    Desgined and implemented a Coq library to reason about multi smart-contract
interaction following the computational model of the Tezos Blockchain.

    \divider

    \cveventpp{Haskell Lola}{}{2019 -- 2020}{\cvreference{\faGithub}{https://gitlab.com/user/repo}\cvreference{\faGlobe}{https://software.imdea.org/hlola/}}
    Designed the core structure implementing HLola.
    Implemented the main abstraction lifting generic types of Haskell
employed as data-theories in Lola.

    \divider

    \cveventpp{QuickFuzz}{}{2016 -- 2017}{\cvreference{\faGithub}{https://github.com/CIFASIS/QuickFuzz}}
        Out-of-the-box fuzzy generation of arbitrary files.
        Implemented Haskell arbitrary instances for arbitrary data
                structures in Haskell.

    \divider

    \cveventpp{MegaDeTh}{}{2016 -- 2017}{\cvreference{\faGithub}{https://github.com/CIFASIS/megadeth}}
    Aggressive and generic instance derivation for recursive and mutually
dependent Haskell data-types.
    Implemented entirely in Template Haskell.

    \divider
    \cveventpp{EasyCrypt}{}{2013}{\cvreference{\faGithub}{https://github.com/EasyCrypt/easycrypt}}
    Implemented tactics to simplify equations over rings and fields.
    Designed and implemented automatic optimistic sampling.
% ----- PROJECTS -----
%%% Local Variables:
%%% TeX-master: "ceresa"
%%% End:

      %%
      \switchcolumn
      %%
      %% Right column
      % \vspace{0.3em}
      % ----- ABOUT ME -----
% \cvsection{About Me}
%     \begin{quote}
%         Lorem ipsum dolor sit amet, consectetur adipiscing elit, sed do eiusmod tempor incididunt ut labore et dolore magna aliqua.
%     \end{quote}
% ----- ABOUT ME -----

% ----- EXPERIENCE -----
\cvsection{Experience}
    \cvevent{Postdoc Researcher}{IMDEA Software Institute}{03 2023 -- Ongoing}{Madrid, Spain}
    \begin{itemize}
        \item Studied offline monitoring properties of smart-contracts.
        \item Designed and implemented (in Haskell) \emph{Termina}, a
            \textbf{non}-turing complete language for embedded software.
        \item In progress: developing a new specification language mixing formal
            specification and executable programs, a generalization of Prophecy Variables.
        \item Guided a PhD student and two interns.
    \end{itemize}
    \divider

    \cvevent{Research Programmer}{IMDEA Software Institute}{10 2022 -- 02 2023}{Madrid, Spain}
        Implemented (in Rust) an intermediate layer expanding previous
            Tezos indexer features to monitor Tezos smart-contract with HLola~\cite{HLola}.
    % \begin{itemize}
    %     \item Implemented (in Rust) an intermediate layer expanding previous
    %         Tezos indexer features to monitor Tezos smart-contract with HLola~\cite{HLola}.
    %     \item Second achievement
    %     \item Third achievement
    % \end{itemize}

    \divider

    \cvevent{Visting PhD Student}{IMDEA Software Institute}{08 2020 -- 09 2022}{Madrid, Spain}
    \begin{itemize}
        \item Implemented a new \emph{Coq} library,
            \emph{Multi}~\cite{Ceresa.2022.Multi}, to prove and reason about multi-contract
            interaction inspired on the computational model of Tezos blockchain.
        \item Formalized (in Coq) different execution strategies, inspired by
            Tezos execution model, while introducing online and onchain smart-contract
            monitoring capabilities.
        \item Collabored on the development of a theory to introduce
            online transaction monitors to blockchain
            technologies~\cite{Capretto.2022.TransactionMonitors}.
        \item Collabored on the implementation and development of
            Setchain~\cite{Capretto.2022.Setchain}, a new Byzantine-tolerant distributed
            data-structure.
    \end{itemize}
    \divider

    \cvevent{Undergraduate Intern}{IMDEA Software Institute}{03 2013 -- 09 2013}{Madrid, Spain}
    \begin{itemize}
        \item Expanded the strategy toolbox of
            \emph{EasyCrypt}~\cite{EasyCrypt}, a modern proof assistant for cryptrographers.
        \item Learned about modern cryptography proofs and game simulation-based proofs.
        \item Implemented (in OCaml) two different tactics: field/ring equality
            and automatic optimistic sampling.
    \end{itemize}
% ----- EXPERIENCE -----

\newpage


% ----- Teaching -----
\cvsection{TEACHING}
    \cvevent{Adjunct Professor}{UNR}{2019 -- 2021}{Rosario, Argentina}
    \begin{itemize}
        \item Parallel and Concurrent Programs: I taught the basic
            problems and solutions of concurrent and parallel programming in \emph{C} and \emph{Erlang}.
        \item Compilers Course: I followed closely the book \emph{Modern
                Compilers Implementation in ML} by \emph{Andrew Appel}
            implementing a full-fledged compiler for the language
            \emph{Tiger} without optimizations.
    \end{itemize}
    \divider

    \cvevent{Teacher Assistant}{UNR}{2011 -- 2019}{Rosario, Argentina}
    My responsibilities included: assisting and conducting laboratory
    hours, crafting tests and final projects and grading them.
    \begin{itemize}
        \item {Data Structures and Algorithms in C}
        \item {Compiler Theory and Practice}
        \item {Concurrent Programming in Erlang}
        \item {Introduction to pure programming languages and Haskell}
        \item {Parallel Thinking in Haskell}
        \item {Introduction to Lambda Calculus, STLC, Intermediate functional programming in Haskell}
        \item {Introduction to Intuitionistic Logic}
        \item {Introduction to Generalized Abstract Data Types}
    \end{itemize}
% ----- Teaching -----

%  ----- Courses -----
\cvsection{COURSES}
    \cvevent{\(\lambda\)-Calculus and Reasonable Cost Models}{UBA}{2019}{Buenos Aires, Argentina}

    %%%%%%%%%%%%%%%%%%%%%%%%%%%%%%%%%%%%%%%%%%%%%%%%%
    \cvevent{DeepSpec Summer School\cvreference{\faGlobe}{https://deepspec.org/event/dsss18/}}{Princeton University}{2018}{New Jersey, United States of America}

    %%%%%%%%%%%%%%%%%%%%%%%%%%%%%%%%%%%%%%%%%%%%%%%%%
    \cvevent{Introduction to basic and algebraic Topology}{UNR}{2018}{Rosario, Argentina}

    %%%%%%%%%%%%%%%%%%%%%%%%%%%%%%%%%%%%%%%%%%%%%%%%%
    \cvevent{Haskell's Type System/GHC and its extensions \cvreference{\faGlobe}{https://cs.famaf.unc.edu.ar/~hoffmann/rio18/}}{ECI}{2018}{Río Cuarto,
    Argentina}
    {FlexibleContexts, FlexibleInstances, GADTs, Phantom Types,
    RankNPoly, DataKinds, MultiParamTypeClasses, FunctionalDependencies,
    TypeFamilies.}

    %%%%%%%%%%%%%%%%%%%%%%%%%%%%%%%%%%%%%%%%%%%%%%%%%
    % \cvevent{Quantum Computing Introduction and Foundation of Programming
    %     Languages}{UNR}{2016}{Rosario, Argentina}
    % {Introduction to Quantum Computing and Lambda-Calculus by Alejandro Díaz-Caro.}

    %%%%%%%%%%%%%%%%%%%%%%%%%%%%%%%%%%%%%%%%%%%%%%%%%
    \cvevent{Introduction to Category Theory and Agda}{UNR}{2016}{Rosario, Argentina}
    % { by Mauro Jaskelioff.}

    %%%%%%%%%%%%%%%%%%%%%%%%%%%%%%%%%%%%%%%%%%%%%%%%%
    % \cvevent{Paradox and Infinity }{2015}{MIT}{Edx(Online)}{\href{https://verify.edx.org/cert/b343b0a81d54493289f2b06ca57c4fde}{Approved Certification}.}

    %%%%%%%%%%%%%%%%%%%%%%%%%%%%%%%%%%%%%%%%%%%%%%%%%
    \cvevent{Interactive theorem proving: theory and practice}{UBA}{2014}{Buenos Aires, Argentina}
    {Introduction to Coq.
    Natural deduction, Curry-Howard correspondence and Lambda Calculus.
    \href{http://www.dc.uba.ar/events/eci/2014/cursos/ziliani}{Course Information}.}

    %%%%%%%%%%%%%%%%%%%%%%%%%%%%%%%%%%%%%%%%%%%%%%%%%
    \cvevent{Introduction to Type Theory in Agda \cvreference{\faGlobe}{https://www.cs.nott.ac.uk/~psztxa/rosario/}}{UNR}{2011}{Rosario, Argentina}
    % {Dependent Type Programming with Agda by Thorsten Altenkirch.}
%  ----- Courses -----


%%% Local Variables:
%%% TeX-master: "ceresa"
%%% End:

    \end{paracol}

    %%%%%
    \newpage
    %%%%%
    %% Publications.
    \cvsection{Publications}
    \mynames{Ceresa/Martin,
      Martin/Ceresa,
      Mart{\'\i}n/Ceresa,
      Ceresa/Mart{\'{\i}}n}
      \nocite{*}
    %%%%%
\printbibliography[heading=pubtype,title={\printinfo{\faFile*[regular]}{Journal Articles}},type=article]

\divider

\printbibliography[heading=pubtype,title={\printinfo{\faUsers}{Conference Proceedings}},type=inproceedings]

\end{document}
