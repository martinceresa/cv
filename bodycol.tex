% ----- ABOUT ME -----
% \cvsection{About Me}
%     \begin{quote}
%         Lorem ipsum dolor sit amet, consectetur adipiscing elit, sed do eiusmod tempor incididunt ut labore et dolore magna aliqua.
%     \end{quote}
% ----- ABOUT ME -----

% ----- EXPERIENCE -----
\cvsection{Research Experience}
    \cvevent{Postdoc Researcher}{IMDEA Software Institute}{03 2023 -- Ongoing}{Madrid, Spain}
    \begin{itemize}
        \item Ongoing research: formalizing Fraud Proof games in \emph{Lean}.
        \item Studied offline monitoring properties of smart-contracts.
        \item Designed and implemented (in Haskell) Termina, a
            \textbf{non}-turing complete language for embedded software.
        % \item In progress: developing a new specification language mixing formal
        %     specification and executable programs, a generalization of Prophecy Variables.
        \item Developed a simulation environment for the Tezos blockchain smart contracts in \emph{Go}.
        \item Guided a PhD student and three interns.
    \end{itemize}

    \divider

    \cvevent{Research Programmer}{IMDEA Software Institute}{10 2022 -- 02 2023}{Madrid, Spain}
        Implemented (in Rust) an intermediate layer expanding previous
            Tezos indexer features to monitor Tezos smart-contract with HLola.
    % \begin{itemize}
    %     \item Implemented (in Rust) an intermediate layer expanding previous
    %         Tezos indexer features to monitor Tezos smart-contract with HLola~\cite{HLola}.
    %     \item Second achievement
    %     \item Third achievement
    % \end{itemize}

    \divider

    \cvevent{Visiting PhD Student}{IMDEA Software Institute}{08 2020 -- 09 2022}{Madrid, Spain}
    \begin{itemize}
        \item Implemented a Coq library,
            \emph{Multi}, to prove and reason about multi-contract
            interaction inspired on the computational model of the Tezos blockchain.
        \item Formalized (in Coq) different execution strategies, inspired by
            Tezos execution model, while introducing online and onchain smart-contract
            monitoring capabilities.
        \item Participated in the development of a theory to introduce
            online transaction monitors to blockchain
            technologies.
        \item Participated in the implementation and development of
            Setchain, a new Byzantine-tolerant distributed
            data-structure.
    \end{itemize}
    \divider

    \cvevent{Undergraduate Intern}{IMDEA Software Institute}{03 2013 -- 09 2013}{Madrid, Spain}
    \begin{itemize}
        \item Expanded the strategy toolbox of
            EasyCrypt, a modern proof assistant for cryptrographers.
        \item Learned about modern cryptography proofs and game simulation-based proofs.
        \item Implemented (in OCaml) two different tactics: field/ring equality
            and automatic optimistic sampling.
    \end{itemize}
% ----- EXPERIENCE -----

\newpage


% ----- Teaching -----
\cvsection{TEACHING Experience}
    \cvevent{Adjunct Professor}{UNR}{2019 -- 2021}{Rosario, Argentina}
    \begin{itemize}
        \item Parallel and Concurrent Programs: I taught the basic
            problems and solutions of concurrent and parallel programming in C and Erlang.
        \item Compilers Course: I followed closely the book \emph{Modern
                Compilers Implementation in ML} by \emph{Andrew Appel}
            implementing a full-fledged compiler for the language
            \emph{Tiger} without optimizations.
    \end{itemize}
    \divider

    \cvevent{Teacher Assistant}{UNR}{2011 -- 2019}{Rosario, Argentina}
    My responsibilities included: assisting and conducting laboratory
    hours, crafting tests and final projects and grading them.
    \begin{itemize}
        \item {Data Structures and Algorithms in C}
        \item {Compiler Theory and Practice}
        \item {Concurrent Programming in Erlang}
        \item {Introduction to pure programming languages and Haskell}
        \item {Parallel Thinking in Haskell}
        \item {Introduction to Lambda Calculus, STLC, Intermediate functional programming in Haskell}
        \item {Introduction to Intuitionistic Logic}
        \item {Introduction to Generalized Abstract Data Types}
    \end{itemize}
% ----- Teaching -----

%  ----- Courses -----
\cvsection{COURSES}
    \cvevent{\(\lambda\)-Calculus and Reasonable Cost Models\cvreference{\faGlobe}{https://eci2019.dc.uba.ar/Accattoli.pdf}}{UBA}{2019}{Buenos Aires, Argentina}

    \divider

    %%%%%%%%%%%%%%%%%%%%%%%%%%%%%%%%%%%%%%%%%%%%%%%%%
    \cvevent{DeepSpec Summer School\cvreference{\faGlobe}{https://deepspec.org/event/dsss18/}}{Princeton University}{2018}{New Jersey, United States of America}

    \divider

    %%%%%%%%%%%%%%%%%%%%%%%%%%%%%%%%%%%%%%%%%%%%%%%%%
    \cvevent{Introduction to basic and algebraic Topology}{UNR}{2018}{Rosario, Argentina}

    \divider

    %%%%%%%%%%%%%%%%%%%%%%%%%%%%%%%%%%%%%%%%%%%%%%%%%
    \cvevent{Haskell's Type System/GHC and its extensions \cvreference{\faGlobe}{https://cs.famaf.unc.edu.ar/~hoffmann/rio18/}}{ECI}{2018}{Río Cuarto,
    Argentina}
    {FlexibleContexts, FlexibleInstances, GADTs, Phantom Types,
    RankNPoly, DataKinds, MultiParamTypeClasses, FunctionalDependencies,
    TypeFamilies.}

    \divider

    %%%%%%%%%%%%%%%%%%%%%%%%%%%%%%%%%%%%%%%%%%%%%%%%%
    % \cvevent{Quantum Computing Introduction and Foundation of Programming
    %     Languages}{UNR}{2016}{Rosario, Argentina}
    % {Introduction to Quantum Computing and Lambda-Calculus by Alejandro Díaz-Caro.}

    %%%%%%%%%%%%%%%%%%%%%%%%%%%%%%%%%%%%%%%%%%%%%%%%%
    \cvevent{Introduction to Category Theory and Agda\cvreference{\faGlobe}{https://www.fceia.unr.edu.ar/~mauro/}}{UNR}{2016}{Rosario, Argentina}
    % { by Mauro Jaskelioff.}

    \divider

    %%%%%%%%%%%%%%%%%%%%%%%%%%%%%%%%%%%%%%%%%%%%%%%%%
    % \cvevent{Paradox and Infinity }{2015}{MIT}{Edx(Online)}{\href{https://verify.edx.org/cert/b343b0a81d54493289f2b06ca57c4fde}{Approved Certification}.}

    %%%%%%%%%%%%%%%%%%%%%%%%%%%%%%%%%%%%%%%%%%%%%%%%%
    \cvevent{Interactive Theorem Proving\cvreference{\faGlobe}{https://eci2014.dc.uba.ar/ziliani.html}}{UBA}{2014}{Buenos Aires, Argentina}
    {Introduction to Coq.
    Natural deduction, Curry-Howard correspondence and Lambda Calculus.}

    \divider

    %%%%%%%%%%%%%%%%%%%%%%%%%%%%%%%%%%%%%%%%%%%%%%%%%
    \cvevent{Introduction to Type Theory in Agda \cvreference{\faGlobe}{https://www.cs.nott.ac.uk/~psztxa/rosario/}}{UNR}{2011}{Rosario, Argentina}
    % {Dependent Type Programming with Agda by Thorsten Altenkirch.}
%  ----- Courses -----


%%% Local Variables:
%%% TeX-master: "ceresa"
%%% End:
