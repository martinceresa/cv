\documentclass[11pt,a4paper,sans]{moderncv}
% possible options include font size ('10pt', '11pt' and '12pt'), paper size
% ('a4paper', 'letterpaper', 'a5paper', 'legalpaper', 'executivepaper' and
% 'landscape') and font family ('sans' and 'roman')

% moderncv themes
\moderncvstyle{classic}                        % style options are 'casual' (default), 'classic', 'oldstyle' and 'banking'
\moderncvcolor{green}                          % color options 'blue' (default), 'orange', 'green', 'red', 'purple', 'grey' and 'black'
%\renewcommand{\familydefault}{\sfdefault}    % to set the default font; use '\sfdefault' for the default sans serif font, '\rmdefault' for the default roman one, or any tex font name
%\nopagenumbers{}                             % uncomment to suppress automatic page numbering for CVs longer than one page
\definecolor{greencv}{rgb}{0.55,0.85,0.35}
\definecolor{greycv}{rgb}{0.75,0.75,0.75}

% adjust the page margins
\usepackage[scale=0.75]{geometry}
%\setlength{\hintscolumnwidth}{3cm}           % if you want to change the width of the column with the dates
%\setlength{\makecvtitlenamewidth}{10cm}      % for the 'classic' style, if you want to force the width allocated to your name and avoid line breaks. be careful though, the length is normally calculated to avoid any overlap with your personal info; use this at your own typographical risks...

\usepackage[utf8]{inputenc}
\usepackage{amsbsy}
\usepackage{amsfonts}
\usepackage{amsmath}

% personal data
\firstname{Martín}
\familyname{Ceresa}
% \address{Moreno 380 6 `A'}{2000 Rosario}
\mobile{+34~684~319338}
\email{martinceresa@gmail.org}
\homepage{ceresa.ar}



\title{Curriculum Vitae}
%\extrainfo{additional information} 
%\quote{Some quote}                

% bibliography with mutiple entries
\usepackage{multibib}
\newcites{conferences}{{Papers in refereed conferences}}
\newcites{journals}{{Journals}}
%----------------------------------------------------------------------------------
%            content
%----------------------------------------------------------------------------------
\AtBeginDocument{%
    \hypersetup{colorlinks,urlcolor=blue}
}
%%%%%%%%%%%%%%%%%%%%%%%%%%%%%%%%%%%%%%%%
%% Commands
\newcommand\burl[1]{[\url{#1}]}
%%%%%%%%%%%%%%%%%%%%%%%%%%%%%%%%%%%%%%%%
\begin{document}
%-----       resume       ---------------------------------------------------------
\makecvtitle%
% \today
\renewcommand{\listitemsymbol}{-~}            % change the symbol for lists
% \section{Personal Information}
% \cvlistitem{Age: 29}
% \cvlistitem{Date of birth: March 1st, 1990}
% \cvlistitem{Place of birth: Rosario, Santa Fe, Argentina}
% \cvlistitem{Nationalities: Argentinian, Italian}
% \cvlistitem{Argentinian Passport: 35130605N}
% \cvlistitem{Italian Passport: YA4033590}
% \cvlistitem{Marital Condition: Single}
\section{Education}
\cventry{2015--2023}{PhD Student in Computer Science}{National University of
  Rosario}{Rosario}{Santa Fe, Argentina}{}
\cvline{Thesis}{\emph{Effectful Improvement Theory}}
\cvline{Supervisor}{Dr. Mauro Jaskelioff}
% \cvline{description}{Cost relations to prove program optimizations in functional
% languages with algebraic effects.}
% \cvline{description}{Improvement Theory is theory based on a cost relation of programs where its basic
%   observation is related to improve certain resource consumption, such as
%   energy or evaluation steps. Since it is based on exact costs, it is extremely
%   useful to prove compilers optimization.
%   There is a fault that the theory depends on features of the evaluation of such
%   languages, therefore, if someone wants to use such theory for a new language,
%   she would have to derive a lot of theorems and lemmas before the theory
%   became useful.
%   This Thesis presents a way to derive Improvement Theories for different
%   languages given its interpreters have certain properties such as modularity and
%   monotonic cost consumption.}
% {I have a
% scholarship completely funded by \href{http://www.conicet.gov.ar/}{CONICET}, the national
% research council
% and I am currently working at
% \href{http://www.cifasis-conicet.gov.ar/}{CIFASIS}.}
\cventry{2008--2015}{Computer Science Degree}{National University
  of Rosario}{Rosario}{Santa Fe, Argentina}{
  Grade Point Average: 9.60. Top Student Award Class 2014--2015.}  % arguments 3
% \cvitem{}{Degree's Thesis}
\cvline{Thesis}{\emph{Simulation of Parallel Programs in Haskell}}
\cvline{Advisors}{Dr. Mauro Jaskelioff and Dr. Exequiel Rivas}
% \cvline{description}{%
%     The use of the basic parallel combinators in \emph{Haskell} produces a certain
% effect in run-time, which is the parallelization of our program. Since that
% effect is not directly observable, we can try and guess what happened using
% tools like \emph{ThreadScope}. However, one can not match the behavior of the
% program's execution and its source code. Therefore, in this work we designed an
% EDSL around the basic combinators to produce a \emph{parallel structure} of
% programs, a data type that reassembles how the parallel combinators were used in
% the source code, so one can see, and analyze the inherent parallel behavior of
% our programs. The aim of this work was the development of a tool to make
% parallelism observable, so one can use this information to help undergraduate
% students get a better understanding, or to study the parallelism of our
% programs.
% } 

\section{Experience}
\subsection{Postdoc Researcher}
\cventry{03/2023--Today}{Instituto Madrile\~no de Estudios
Avanzado}{Madrid}{Espa\~na}{}{%
Offline monitoring properties of smart-contracts.
Developing \emph{Termina}, a \textbf{non}-turing complete language designed to
write safe programs for satellites.
Theoretical research: developing a specification language breaking the barrier
between formal specifications and executable programs.
}{}
\subsection{Research Programmer}
\cventry{10/2022--02/2023}{Instituto Madrile\~no de Estudios
Avanzado}{Madrid}{Espa\~na}{}{%
Development of offline monitoring indexing providing a bridge between
HLola~\cite{HLola} and the Tezos Blockchain~(\url{https://tezos.com/}).
}{}
\subsection{Visiting PhD Student}
\cventry{09/2021-09/2022}{Instituto Madrile\~no de Estudios
Avanzado}{Madrid}{Espa\~na}{}{%
Mechanization of previous results in \emph{Coq} resulting in a new framework
called \emph{Multi} implemented to prove and study properties of multi-contract
interaction and different blockchain features\cite{Ceresa.2022.Multi}.
%
Collaboration on the development of online transaction
monitors~\cite{Capretto.2022.TransactionMonitors} and setchain~\cite{Capretto.2022.Setchain}.
}{}
\subsection{Intern}
\cventry{08/2020-02/2021}{Instituto Madrile\~no de Estudios
Avanzado}{Madrid}{Espa\~na}{}{%
Formalizing and implementing several blockchain execution strategies for
online and onchain smart contract monitoring.
%
We also explored:
\begin{itemize}
  \item solutions to offline monitoring blockchain and blockchain in order to
devise new high-level query language;
  %
  \item how to retrieve and analyze ephemeral data employed by the blockchain
ecosystem, e.g. the mempool, branch-blocks, for off-line monitoring implementations;
  %
  \item solutions to mitigate front-running attacks
\end{itemize}
}{}
\cventry{03/2013--09/2013}{Instituto Madrile\~no de Estudios
Avanzado}{Madrid}{Espa\~na}{}{%
Building a tool for synthesizing and automatic proving cryptographic
constructions for \href{www.easycrypt.info}{EasyCrypt}.
%
Consisted of: Learning about Modern Cryptography proofs, game simulation based
proofs.
Tactics implemented: field/ring equality, and automatic optimistic sampling
tactic.
Language employed: OCaml.
}{}

\subsection{Projects}
\cvline{Termina}{Transpiler from Termina language to C.}
\cvline{Multi}{Coq Multi Smart-Contract Interaction library exploring different
smart-contracts execution models.}
\cvline{HLola}{Haskell Lola
  implementation\burl{https://github.com/imdea-software/hlola}. Remote
  collaboration with the Stream Runtime Verification group at IMDEA-Software.} 
\cvline{QuickFuzz}{An automatic random fuzzer for common file formats. Open
Source project. Haskell Code:
\burl{https://github.com/CIFASIS/QuickFuzz}.}
\cvline{MegaDeTh}{An experimental automatic and \textbf{aggressive} instance
derivator in Haskell using Template Haskell. Open Source project. Haskell Code:
\burl{https://github.com/CIFASIS/megadeth}}

% \newpage

\subsection{Teaching}
\cventry{2019--2021}{Adjunct Professor}{National University of Rosario}{Rosario}{Argentina}{}
% \cventry{2019--}{Adjunct Professor}{Compilers}{National University of
%   Rosario}{}
\cvitem{Parallel and Concurrent Programs}{%
  I taught the basic mechanisms and problems of concurrent
  and parallel programming. It is taught in \emph{C} and \emph{Erlang}.}
\cvitem{Compilers Course}{This course is based on the implementation
of a full \emph{Tiger's} compiler, presented by Andrew Appel's book
\emph{Modern Compilers Implementation in ML}, without optimizations.}
\cventry{2011 - 2019}{Teacher Assistant}{National University of Rosario}{Rosario}{Argentina}{}
\cvitem{}{Worked in courses that involved:}
\cvlistitem{Data Structures and Algorithms in C}
\cvlistitem{Compiler Theory and Practice}
\cvlistitem{Concurrent Programming in Erlang}
\cvlistitem{Introduction to pure programming languages and Haskell}
\cvlistitem{Parallel Thinking in Haskell}
\cvlistitem{Introduction to Lambda Calculus, STLC, Intermediate functional programming in Haskell}
\cvlistitem{Introduction to Intuitionistic Logic}
\cvlistitem{Introduction to Generalized Abstract Data Types}
\cvitem{}{My responsibilities included: assisting and conducting practical exercising
  hours, proposing exams and final projects, and grading practice exams.}
\cventry{09/2015--02/2016}{Teacher}{Introduction to Mathematics}{National
University of Rosario}{}{Introductory course for pre-university students. I
presented an introduction to basic mathematics, as well as assist them in solving
exercises.}

\newpage
\subsection{Programming Competitions}
\cventry{2009}{3rd prize in the First Competitive Programming Encounter}{Santa Fe}{Argentina}{}{}{}
\cventry{2009}{Participant in South America Regional ACM-ICPC}{Buenos Aires}{Argentina}{}{}{}

\section{Skills}
\cvline{Languages}{Haskell (proficient), Agda (experienced), SML (prior experience), OCaml
  (prior experience), C (prior experience), Erlang (prior experience) }
\cvline{Proof Assistants}{Coq (experienced)}
\cvline{Specification Languages}{Z, CSP, TLA, Statecharts}
\cvline{Technologies}{ViM, Git, \LaTeX, Markdown, Org, GNU/Linux}
\cvline{Haskell Environment}{Property Based Testing (QuickCheck),
Metaprogramming (Template Haskell), XMonad, stack, cabal}
% \cvline{Others}{R, prolog, Scilab}

\subsection{Languages}
\cvitemwithcomment{Spanish}{Native}{}
\cvitemwithcomment{English}{Advance}{Fluent Written and Spoken}

% \section{Extra Courses}
\subsection{Courses}
\cventry{2019}{\(\pmb{\lambda}\)-Calculus and Reasonable Cost Models}{Buenos
Aires}{Argentina}{}{Introduction to reasonable cost models. Size
explosion. Abstract machines and a proof that the call-by-name weak \(\lambda\)-calculus is
reasonable. The call-by-value case. The open case and its different
presentation.The standardization theorem and the subterm property.}
\cventry{2018}{DeepSpec Summer School}{Princeton, NJ}{}{}{\burl{https://deepspec.org/event/dsss18/}}
\cventry{2018}{Topology II}{National University of Rosario}{Rosario}{Santa Fe,
    Argentina}{Introduction to algebraic Topology.}
\cventry{2018}{Topology I}{National University of Rosario}{Rosario}{Santa Fe,
    Argentina}{Introduction to Topology.}
\cventry{2018}{Haskell's Type System/GHC and its extensions}{Escuela de Verano
  de Ciencias Informáticas}{Rio Cuarto}{Cordoba,
  Argentina}{FlexibleContexts, FlexibleInstances, GADTs, Phantom Types,
RankNPoly, DataKinds, MultiParamTypeClasses, FunctionalDependencies,
TypeFamilies. \href{https://cs.famaf.unc.edu.ar/~hoffmann/rio18/}{Course Information}}
\cventry{2016}{Quantum Computing Introduction and Foundation of Programming
    Languages}{National University of Rosario}{Rosario}{Santa Fe,
    Argentina}{Introduction to quantum computing by Alejandro Díaz-Caro.}
\cventry{2016}{Programming with Categories}{National University of Rosario}{Rosario}{Santa Fe, Argentina}
{Introduction to Category Theory and Agda by Mauro Jaskelioff.}
\cventry{2014}{Interactive theorem proving: theory and practice}{University of
Buenos Aires}{Buenos Aires}{Buenos Aires, Argentina}{Introduction to Coq.
Natural deduction, Curry-Howard correspondence and Lambda Calculus.
\href{http://www.dc.uba.ar/events/eci/2014/cursos/ziliani}{Course Information}.}
\cventry{2011}{Type Theory}{National University of Rosario}{Rosario}{Santa Fe, Argentina}{Dependent Type Programming with Agda by Thorsten Altenkirch.}

\subsection{Online Courses}
\cventry{2015}{Paradox and Infinity}{Agustín
Rayo}{MIT}{Edx}{\href{https://verify.edx.org/cert/b343b0a81d54493289f2b06ca57c4fde}{Approved
Certification}.}

% Publications from a BibTeX file using the multibib package
\section{Publications}
%%%%%%%%%%%%%%%% Journals
\bibliographystylejournals{plain}
\nocitejournals{QF:Jorn}
\nocitejournals{CeresaEffectful}
\bibliographyjournals{journals}                   % 'publications' is the name of a BibTeX file
%%%%%%%%%%%%%%%% Conferences
\bibliographystyleconferences{plain}
\nociteconferences{HLola}
\nociteconferences{QF:Haskell}
\nociteconferences{Ceresa.2022.Multi}
\nociteconferences{Capretto.2022.Setchain}
\nociteconferences{Capretto.2022.TransactionMonitors}
\bibliographyconferences{conferences}                   % 'publications' is the name of a BibTeX file

% \subsection{Personal Skills}
% \cvitem{}{An active and continuous learner}
% \cvitem{}{Excellent predisposition to challenges}

% \clearpage
%-----       letter       ---------------------------------------------------------
% recipient data
%\recipient{}{Company, Inc.\\123 somestreet\\some city}
%\date{January 01, 1984}
%\opening{Dear Sir or Madam,}
%\closing{Yours faithfully,}
%\enclosure[Attached]{curriculum vit\ae{}}     % use an optional argument to use a string other than "Enclosure", or redefine \enclname
%\makelettertitle

%\[ e=\lim_{n \to \infty} \left(1+\frac{1}{n}\right)^n \]
% \section{Research Statement}

% One of my main interest is functional programming languages.
% I think that we can use functional programming as a tool to
% understand how programming works, in particular pure functional languages give
% us the ways to explicitly observe what is going on, understand what we are
% observing, and build new concepts. It gives us the tools to think, create and
% test our ideas based in powerful mathematical concepts.

% I am working on a verified library in Agda to analyze costs models based on
% Category Theory. The main idea is to find a bridge between Improvement Theory
% developed by Andrew Sands and Functorial Semantics developed by Daniele Turi and
% Gordon Plotkin where I would like to build an improvement theory based on GSOS
% rules.

% My interest includes: functional programming, compiler theory, denotational
% semantics, generic programming, static analysis and type theory.
% I like Category Theory, abstract mathematical tools,
% and love Haskell.
%\makeletterclosing

\end{document}
%% end of file `template.tex'.

% \subsection{Have Experience With}
% \cvline{Formal Specification Languages}{Z, CSP, TLA/TLA+, Statecharts}
% \cvline{Database Theory}{SQL, Entity-Relationship Model, Relational Algebra}
% \cvline{Technologies}{Emacs}
% \cvline{Languages}{Erlang, OCaml}

%\section{Extra 2}
%\cvlistdoubleitem{Item 1}{Item 4}
%\cvlistdoubleitem{Item 2}{Item 5\cite{book1}}
%\cvlistdoubleitem{Item 3}{}
